\chapter{Introduction}

\paragraph{}
Massive scale datasets are becoming increasingly common today. Growth of the number of users who are actively using digital devices connected to the internet has vastly affected this phenomenon. Also, there lies an interest in researchers to solve the problems which involve large datasets. Most of these datasets could be mapped to graphs to extract useful information, giving rise to the need for processing massive scale graphs. There are many practical scenarios where massive scale graphs are applied such as social networks, network traffic data and road networks.

\paragraph{}
It is much easier to work with graphs when they are static and small. However most of the natural graphs that are being encountered in the real world are dynamic. It becomes increasingly complex to handle the graph as the velocity with which its edges gets updated increases. Large scale dynamic natural graphs are used by many companies today. Google uses the PageRank algorithm\cite{brin_anatomy_1998} to map the links between the web pages. And Facebook has a massive graph with trillions of edges\cite{ching_one_2015}, depicting the connections of each user on the platform. 

\paragraph{}
With the size of the massive scale graphs, it is difficult to evaluate the properties of them even after partitioning into multiple nodes. The graphs have to be summarized so that important information regarding the underlying dataset can be inferred easily.

\paragraph{}
Being applied in a wide range of industrial and research applications, realtime property evaluation of streaming and dynamic natural graphs is a critical requirement in many scenarios. Graph summarization plays a big role in this as it reduces the computational resources required to evaluate the properties in a rather massive scale streaming graph. It would be beneficial for a number of sectors in the process of summarizing streaming graphs were made efficient.