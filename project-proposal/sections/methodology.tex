\section{Methodology}

\paragraph{}
The project will deal with large scale streaming and dynamic graphs. 
Therefore the first step would be obtaining available dataset of large 
natural graphs. One type of dataset that we will mainly focus on is the 
interactions of users in a social network. There are existing crawlers 
which are able to scrape things like tweets and facebook posts. We will also 
use a web crawler as suggested in the original work, to build a graph while 
the crawler is traversing through the web. “This is an unbounded graph since 
we do not know how large the graph would be and the graph keeps building while 
the crawler keeps traversing.”[11] And when it is within the acceptable 
boundary conditions of the research, we will use synthetic graphs to test 
out the algorithms and various corner cases. 

\paragraph{}
Then we will spend time on preparing an adequate environment for the sketching 
and partitioning operations to be run. There exists number of graph frameworks 
to facilitate graph computations such as Pregel, GraphLab, 
Distributed GraphLab, Giraph, Giraph++.  And data streaming frameworks 
like Apache Flink will be studied and used for the development purposes.  

\paragraph{}
Many streaming graph partitioning and sketching algorithms proposed by the 
previous researches haven’t made the codebase available along with the paper. 
So these will be re-implemented to be run on our environment of choice. In 
the researches where the sketching algorithms were proposed, there were 
evaluated against different types of queries, i.e edge queries, node queries 
and path queries. Reevaluating those with a similar set of parameters will 
help us in clearly differentiating the strengths and weaknesses of each 
sketching method. 

\paragraph{}
Hereafter, the implemented sketching mechanisms will be tested in evaluating 
other graph properties like KNN and K-furthest neighbor. There are various 
evaluating techniques proposed by the previous researches in evaluating the 
sketching mechanisms such as Average Relative Error and Number of Effective 
Queries\cite{kumarage_efficient_2017}. And other resource factors such as memory 
usage will also be measured in each test. 

\paragraph{}
After proper evaluation of the strengths and weaknesses of existing methods 
with respect to different graph properties, we will research on improving those 
methods. And we will attempt at proposing how which technique performs better 
when evaluating each graph property. 

\paragraph{}
In the final phase of the project, we will focus on evaluating the models on 
top of a parallel framework as the future work of the original research 
suggests\cite{kumarage_efficient_2017}. And then we will re-evaluate the proposed 
improvements on top of the parallel framework to report any performance 
gained in running them parallelly. 
