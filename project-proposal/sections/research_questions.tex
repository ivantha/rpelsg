\section{Research Questions}

\paragraph{}
Restating the aforementioned, there already exists some techniques of
summarizing streaming graphs such as gSketch, GMatrix and TCM. And the 
efficiency of these methods have been also mentioned in the referenced 
texts with respect to various queries, i.e, edge queries, node queries 
and path queries. Despite the improvements of the recent summarization 
methods like TCM, all these techniques pose tradeoffs with regard to 
evaluating various graph properties and only a small portion of them have 
been tested in the original research work. Therefore the performance of 
these summarization techniques must be compared against each other while 
evaluating other graph properties such as  KNN and K-furthest neighbor. 
Much research has been remaining on optimizing the streaming graph summarization 
techniques with respect to each of those graph properties. 

\paragraph{}
In the original work proposed through “An efficient query platform for 
streaming and dynamic natural graphs”\cite{kumarage_efficient_2017}, which 
was the initial motivation for this research, states future work as, 
“The automatic sketch creation mechanism should be improved with identifying 
better heuristics for deciding when to created a new sketch. And also should 
identify if any better mechanisms for creating and updating new sketches exists”, 
hinting that much work has to be done with regard to automatic sketch creation in 
streaming graph summarization. Furthermore it adds that 
“This is an embarrassingly parallel query framework model, therefore 
the model should be evaluated on top of a parallel framework. And also, 
new metrics has to be defined for thorough evaluation”. This suggests that 
the proposed solution has yet to be implemented parallely in a suitable 
environment and be evaluated. 

\paragraph{}
To summarize the research questions, 
\begin{itemize}
    \item How to optimize the streaming graph sketching process such that 
        graph properties could be identified in realtime?
    \item How to modify the previous work\cite{kumarage_efficient_2017} such that
        it runs on a parallel query framework?
\end{itemize}