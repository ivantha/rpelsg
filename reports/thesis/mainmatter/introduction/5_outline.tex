\section{Outline of the Dissertation}

\paragraph{}
This dissertation spans across the following 6 chapters.

\begin{enumerate}
    \item \textbf{Introduction}\\
          Chapter 1 gives an introduction to the research work carried out through several sections. First, it describes the background of the research. Then it establishes the motive for the research through the sections, research problem and research questions, aims and objectives and the justification for the research. The introduction chapter also includes the research methodology which describes the process of data collection and the delimitations of the scope of the research.

    \item \textbf{Literature review}\\
          Chapter 2 lays out a comprehensive review done on the related work of this research. It includes an introduction to the graphs and their representations, streaming graphs and graph summarization. Here much emphasis has been given for the streaming graph summarization as it is more closely aligned with the topic of the research. In this section, a summary of each streaming graph summarization technique that will be re-implemented and evaluated in chapter 5 and chapter 6 is presented.

    \item \textbf{Research design}\\
          Chapter 3 consist of the research design. In this chapter, the overall architecture of the test suite is explained while focusing on its relevance in answering the research questions in \autoref{section:research_questions}. Furthermore, all the datasets that will be used for the benchmarking purposes will be mentioned along with their origin and the basic properties such as the number of edges in the graph. The ethical implications of using the aforementioned datasets will be discussed as well.

    \item \textbf{Implementation}\\
          Chapter 4 will explain the implementation details of the test suits in a much granular level. It will also address the issues that pertained to the implementation and the workaround the were followed in order to solve them. However, this chapter would not include the full codes for the implemented algorithms; which will appear in the \autoref{appendix:codes} of this dissertation.

    \item \textbf{Results and evaluation}\\
          Chapter 5 contain the results obtained after benchmarking the implemented sketches against the selected datasets.

    \item \textbf{Conclusion}\\
          Chapter 6 discuss how the results mentioned in Chapter 5 has addressed the research questions. An emphasis will be given in highlighting the contribution of this research work to the scientific world. Implications of further research will also be discussed in this chapter.
\end{enumerate}