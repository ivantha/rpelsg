\section{Introduction}

\paragraph{}
This chapter will include the results of all the benchmarking tests that were done using the test suite in \autoref{section:test_suite}. Different sizes of each sketch were considered in benchmarking to ensure that the identified qualities were retained with the increase/decrease of the allocated memory. The proposed solution, Alpha, has been compared against the existing graph summarization sketches; CountMin, gSketch, TCM and GMatrix under the following criteria, which will be explained in detail in the subsequent sections.

\begin{enumerate}
    \item Build-time test
    \item Average Relative Error of edge queries
    \item Number of Effective Queries
    \item k-Top heavy nodes
    \item k-Top heavy edges
    \item Degree distribution
    \item Edge-weight distribution
    \item Edge insertion time
          % \item Clustering
          % \item PageRank
\end{enumerate}

\paragraph{}
Restating what is already mentioned in the \autoref{section:design}, the tested sketches can be mainly categorized into two classes.

\begin{enumerate}
    \item The sketches which support only the edge frequency queries i.e CountMin and gSketch.
    \item The sketches which support many graph queries in general i.e TCM, GMatrix, Alpha
\end{enumerate}

\paragraph{}
The objectives of this chapter are,

\begin{enumerate}
    \item To compare the benchmarking results of the sketching techniques in order to identify the strengths and weaknesses of each sketch in different scenarios.
    \item To compare the proposed algorithm, Alpha, against the sketching techniques of its own class; TCM and GMatrix.
\end{enumerate}

\paragraph{}
We refrain from a direct comparison between Alpha and the edge frequency sketches in many cases as they vary highly in functionality.