\begin{abstract}
    The amount of collected information on data repositories has vastly increased with the advent of the internet. It has become increasingly complex to deal with these massive data streams due to their sheer volume and the throughput of incoming data. Many of these data streams are mapped into graphs, which helps discover some of their properties. However, due to the difficulty in processing massive streaming graphs, they are summarized such that their properties can be approximately evaluated using the summaries. gSketch, TCM, and gMatrix are some of the major streaming graph summarization techniques. Our primary contribution is devising kMatrix, which is much more memory efficient than existing streaming graph summarization techniques. We achieved this by partitioning the allocated memory using a sample of the original graph stream. Through the experiments, we show that kMatrix can achieve a significantly less error for the queries using the same space as that of TCM and gMatrix.  
\end{abstract}