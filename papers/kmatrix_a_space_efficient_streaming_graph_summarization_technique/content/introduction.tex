\section{Introduction}

Massive-scale datasets are becoming increasingly common today. The growth of the number of users who are actively using digital devices connected to the internet has vastly affected this phenomenon. Also, there lies an interest in researchers to solve the problems which involve large datasets. Most of these datasets could be mapped into graphs to extract useful information, giving rise to the need for processing massive scale graphs. There are many practical scenarios where massive scale graphs are applied such as social networks, network traffic data, and road networks. Large scale dynamic natural graphs are used by many companies today. Google uses the PageRank algorithm\cite{brin_anatomy_1998, page_pagerank_nodate} to map the links between the web pages. Facebook has a massive graph with trillions of edges\cite{ching_one_2015}, depicting the interactions of each user on the platform.

It is much easier to work with graphs when they are static and small. However, most of the natural graphs that are being encountered in the real world are dynamic. It becomes increasingly complex to handle the graph as the velocity with which its edges get updated increases. Determining the properties of streaming graphs is a relatively strenuous task than static graphs as they are continually evolving. Thus the traditional graph algorithms cannot be run on streaming graphs due to their dynamic nature. 
% Getting a static snapshot of a streaming graph at a specific timestamp and running conventional graph algorithms on it is one way of addressing this issue. However, this may not be a suitable remedy as the time taken to process the graph’s massive snapshot can render the result less valuable in a time-critical scenario. 
High throughput of update queries requires any other types of queries to be run efficiently and as fast as possible in an unblocking manner. Therefore if there is a need for processing the graph while streaming, separate streaming graph algorithms have to be devised\cite{mcgregor_graph_2014}.   

Since real-world streaming graphs could grow very large in size, they are often stored as partitions in different machines over a network rather than in a single location. It is difficult to evaluate the properties of a graph with high volume and throughput even after the partitioning process, as the whole graph would have to be processed despite the partitioning. Graph summarization is a technique used in dealing with these massive graphs taking the limitations mentioned above into account so that important information regarding the underlying dataset can be inferred easily. In graph summarization, we reduce the complexity of a graph while retaining only some of its properties. These summaries often incur an error when queried due to the loss of information. When the same algorithm is executed on a graph summary and its original graph, the two results are expected to be approximately equal. Here, the error depends on the compression ratio and various other factors. This tradeoff in accuracy is usually worth it for real-world graphs such as social networks when considering the computational cost incurred in obtaining exact answers. Most of the time, the cost of obtaining an exact solution is so high that it is impossible to do so even if the need arises. 

Being applied in a wide range of industrial and research applications, realtime property evaluation of streaming and dynamic natural graphs is a critical requirement in many scenarios. Graph summarization plays a significant role in this as it reduces the computational resources required to evaluate the properties in a rather massive scale streaming graph. It would be beneficial for many sectors if the process of summarizing streaming graphs were made efficient.

In this work, we propose an improved streaming graph summarization technique; kMatrix. It can outperform the existing state of the art summarization sketches by efficiently using the available memory to answer the queries more accurately. We also show that kMatrix is generally faster than the other sketches in handling the graph streams. Despite the number of methods that have been devised for streaming graph summarization, they still lack the accuracy to be used in most real-world scenarios\cite{kumarage_efficient_2017}. Our motivation in improving the existing sketching techniques lies in increasing the efficiency while maintaining the same resource constraints of the application domains, such as real-time property evaluation of the social networks where streaming graph summarization is critical. 

The remainder of this paper is organized as follows. We explore the related work for this research in Section \ref{sec:related_work}. In Section \ref{sec:methodology} and \ref{sec:implementation}, we will focus on the methodology and the implementation respectively. We will summarize all the results obtained during the experiments in Section \ref{sec:results}. Section \ref{sec:future_work} will address the remaining work to be done before deploying the kMatrix in a real world application. We will conclude the paper in Section \ref{sec:conclusion}, highlighting the importance of this work to the graph summarization domain. 