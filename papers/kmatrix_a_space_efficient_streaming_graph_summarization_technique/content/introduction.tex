\section{Introduction}

Massive-scale datasets are becoming increasingly common today. The growth of the number of users who are actively using digital devices connected to the internet has vastly affected this phenomenon. Also, there lies an interest in researchers to solve the problems which involve large datasets. Most of these datasets could be mapped into graphs to extract useful information, giving rise to the need for processing massive scale graphs. There are many practical scenarios where massive scale graphs are applied such as social networks, network traffic data, and road networks.

It is much easier to work with graphs when they are static and small. However, most of the natural graphs that are being encountered in the real world are dynamic. It becomes increasingly complex to handle the graph as the velocity with which its edges get updated increases. Large scale dynamic natural graphs are used by many companies today. Google uses the PageRank algorithm\cite{brin_anatomy_1998, page_pagerank_nodate} to map the links between the web pages. Facebook has a massive graph with trillions of edges\cite{ching_one_2015}, depicting the interactions of each user on the platform.

With the size of the massive scale graphs, it is difficult to evaluate their properties even after partitioning into multiple nodes. The graphs have to be summarized so that important information regarding the underlying dataset can be inferred easily.

Being applied in a wide range of industrial and research applications, realtime property evaluation of streaming and dynamic natural graphs is a critical requirement in many scenarios. Graph summarization plays a significant role in this as it reduces the computational resources required to evaluate the properties in a rather massive scale streaming graph. It would be beneficial for many sectors if the process of summarizing streaming graphs were made efficient.

In this work, we propose an improved streaming graph summarization technique; kMatrix. It can outperform the existing state of the art summarization sketches by efficiently using the available memory to answer the queries more accurately. We also show that kMatrix is generally faster than the other sketches in handling the graph streams. Despite the number of methods that have been devised for streaming graph summarization, they still lack the accuracy to be used in most real-world scenarios\cite{kumarage_efficient_2017}. Our motivation in improving the existing sketching techniques lies in increasing the efficiency of the application domains, such as real-time property evaluation of the social networks where streaming graph summarization is critical. 