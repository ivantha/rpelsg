\section{Background}

Graphs can be divided into static graphs and streaming (dynamic) graphs. Static graphs do not change while streaming graphs are the ones that get updated over a time interval. These update operations could be insertion or deletion of nodes and edges.

Determining the properties of streaming graphs is a relatively strenuous task than static graphs as they are continually evolving. Thus the traditional graph algorithms cannot be run on streaming graphs due to their dynamic nature. Getting a static snapshot of a streaming graph at a specific timestamp and running conventional graph algorithms on it is one way of addressing this issue. However, this may not be a suitable remedy as the time taken to process the graph’s massive snapshot can render the result less valuable in a time-critical scenario. This process is made even more difficult with the speed with which the graph is being updated. High throughput of update queries requires any other types of queries to be run efficiently and as fast as possible in an unblocking manner. Therefore if there is a need for processing the graph while streaming, separate streaming graph algorithms have to be devised\cite{mcgregor_graph_2014}. 

Since real-world streaming graphs could grow very large in size, they are often stored as partitions in different machines over a network rather than in a single location. It is difficult to evaluate the properties of a graph with high volume and throughput even after the partitioning process, as the whole graph would have to be processed despite the partitioning. Graph summarization is a technique used in dealing with these massive graphs taking the limitations mentioned above into account. 

In graph summarization, we reduce the complexity of a graph while retaining only some of its properties. These summaries often incur an error when queried due to the loss of information. When the same algorithm is executed on a graph summary and its original graph, the two results are expected to be approximately equal. Here, the error depends on the compression ratio and various other factors. This tradeoff in accuracy is usually worth it for real-world graphs such as social networks when considering the computational cost incurred in obtaining exact answers. Most of the time, the cost of obtaining an exact solution is so high that it is impossible to do so even if the need arises. 

Summarizing a graph can have many benefits\cite{liu_graph_2018} apart from the speedup of graph algorithms and queries, such as, reduction of data volume and storage\cite{seo_effective_2018}, visualization\cite{dunne_motif_2013, jin_eco_nodate}, noise elimination\cite{zhang_discovery-driven_2010}, privacy preservation\cite{shoaran_zero-knowledge_2013}. 

Graph summarization has a wide range of industrial and research applications. Some of them are clustering\cite{cilibrasi_clustering_2005}, classification\cite{hutchison_compression_2006}, community detection\cite{chakrabarti_fully_nodate}, outlier detection\cite{smets_odd_2011, akoglu_opavion_2012}, pattern set mining\cite{mampaey_tell_2011} and finding sources of infection in large graphs\cite{prakash_spotting_2012}. Throughout this work, our aim lies in query optimization through graph summarization.

Streaming graph summarization is much more complex than summarizing a static graph due to the constant data flow. Since the underlying graph is updated continuously, the summarization process also has to be done in realtime. Almost any static graph summarization technique can be used with a streaming graph snapshot in a specific timestamp. However, mining information using aggregate time snapshots of data could prove to be a less than ideal solution when considering massive data streams. Thus sophisticated sparsification techniques have to be derived in order to summarize streaming graphs.