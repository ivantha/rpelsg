%% bare_jrnl.tex
%% V1.4b
%% 2015/08/26
%% by Michael Shell
%% see http://www.michaelshell.org/
%% for current contact information.
%%
%% This is a skeleton file demonstrating the use of IEEEtran.cls
%% (requires IEEEtran.cls version 1.8b or later) with an IEEE
%% journal paper.
%%
%% Support sites:
%% http://www.michaelshell.org/tex/ieeetran/
%% http://www.ctan.org/pkg/ieeetran
%% and
%% http://www.ieee.org/

%%*************************************************************************
%% Legal Notice:
%% This code is offered as-is without any warranty either expressed or
%% implied; without even the implied warranty of MERCHANTABILITY or
%% FITNESS FOR A PARTICULAR PURPOSE! 
%% User assumes all risk.
%% In no event shall the IEEE or any contributor to this code be liable for
%% any damages or losses, including, but not limited to, incidental,
%% consequential, or any other damages, resulting from the use or misuse
%% of any information contained here.
%%
%% All comments are the opinions of their respective authors and are not
%% necessarily endorsed by the IEEE.
%%
%% This work is distributed under the LaTeX Project Public License (LPPL)
%% ( http://www.latex-project.org/ ) version 1.3, and may be freely used,
%% distributed and modified. A copy of the LPPL, version 1.3, is included
%% in the base LaTeX documentation of all distributions of LaTeX released
%% 2003/12/01 or later.
%% Retain all contribution notices and credits.
%% ** Modified files should be clearly indicated as such, including  **
%% ** renaming them and changing author support contact information. **
%%*************************************************************************


% *** Authors should verify (and, if needed, correct) their LaTeX system  ***
% *** with the testflow diagnostic prior to trusting their LaTeX platform ***
% *** with production work. The IEEE's font choices and paper sizes can   ***
% *** trigger bugs that do not appear when using other class files.       ***                          ***
% The testflow support page is at:
% http://www.michaelshell.org/tex/testflow/


% Please refer to your journal's instructions for other
% options that should be set.
\documentclass[journal,onecolumn]{IEEEtran}
%
% If IEEEtran.cls has not been installed into the LaTeX system files,
% manually specify the path to it like:
% \documentclass[journal]{../sty/IEEEtran}

\input{package_imports.tex}

% *** Do not adjust lengths that control margins, column widths, etc. ***
% *** Do not use packages that alter fonts (such as pslatex).         ***
% There should be no need to do such things with IEEEtran.cls V1.6 and later.
% (Unless specifically asked to do so by the journal or conference you plan
% to submit to, of course.)


% correct bad hyphenation here
\hyphenation{op-tical net-works semi-conduc-tor}


\begin{document}


\input{heading.tex}


% If you want to put a publisher's ID mark on the page you can do it like
% this:
% \IEEEpubid{0000--0000/00\$00.00~\copyright~2015 IEEE}
% Remember, if you use this you must call \IEEEpubidadjcol in the second
% column for its text to clear the IEEEpubid mark.


% use for special paper notices
%\IEEEspecialpapernotice{(Invited Paper)}


% make the title area
\maketitle


\chapter*{Abstract}

\paragraph{}
The amount of collected data on data repositories have vastly increased with the advent of the internet. It has become increasingly difficult to deal with these massive datasets due to their sheer volume and the throughput of data. Many of these data streams can be mapped into graphs, which in turn helps in discovering some of the underlying properties of the stream. The underlying graph keeps evolving as the data keeps getting streamed. It is difficult to evaluate the properties of these streaming graphs due to their dynamic nature. 

\paragraph{}
These massive streaming graphs can be summarized such that some of their properties can be approximately calculated using the summaries. In many of the real-world scenarios involving large streaming graphs, obtaining an approximate answer is sufficient as the cost of obtaining an exact answer can be too high. CountMin, gSketch, TCM and gMatrix are some of the major streaming graph summarization techniques. 

\paragraph{}
This dissertation explains our contribution to streaming graph property evaluation through summarization. The primary contribution is devising the sketching technique Alpha, which is able to minimize the average relative error of the queries beyond the existing summarization techniques while taking the same amount of memory. Alpha is able to do this through partitioning using a sample of the original graph stream. Our next contribution is conducting a survey on streaming graph summarization techniques by benchmarking the widely used sketches against each other and assessing their strengths and weaknesses against different types of queries. 


% Note that keywords are not normally used for peerreview papers.
\begin{IEEEkeywords}
  streaming graphs, summarization, graph querying.
\end{IEEEkeywords}


% For peer review papers, you can put extra information on the cover
% page as needed:
% \ifCLASSOPTIONpeerreview
% \begin{center} \bfseries EDICS Category: 3-BBND \end{center}
% \fi
%
% For peerreview papers, this IEEEtran command inserts a page break and
% creates the second title. It will be ignored for other modes.
\IEEEpeerreviewmaketitle


\section{Introduction}

\paragraph{}
The implementation of the research mainly consists of two parts.

\begin{enumerate}
    \item The implementation of the test suite which carries out the benchmarking tests on all the sketching algorithms.
    \item The implementation of the Alpha, which is the proposed improvement over the existing sketches according to the research question 2 in \autoref{section:research_questions}.
\end{enumerate}

\paragraph{}
The choice of language for implementation has been Python 3 as its clear an concise syntax allows for more attention to be given to the underlying algorithms rather than the language optimizations. The implementation details of the test suite and Alpha sketch will be discussed in finer details throughout the subsequent sections.

\input{content/sample_content.tex}

\section{Conclusion}

kMatrix is a new streaming graph summarization technique proposed through this research. It can answer queries with a significantly lower average relative error with the same amount of memory compared to the existing state-of-the-art sketching techniques, TCM and gMatrix. We have benchmarked kMatrix using three datasets in different application domains. We believe that the experimental results show the superiority of the proposed solution in comparison to the existing steaming graph summarization techniques. 

\input{content/appendix.tex}

\input{content/acknowledgment.tex}

\input{content/references.tex}

\input{content/biography.tex}

% that's all folks
\end{document}
